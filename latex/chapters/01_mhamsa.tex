%% Chapter 01: M'hamsa
%% Amazing double-page spread layout

\multilingualrecipetitle
    {\texthebrew{מחמסה}}
    {M'hamsa}
    {\textarabic{محمصة}}
    {M'hamsa}

% LEFT PAGE: Etymology, Culture, and Image
\leftpagecontent
    {\textbf{ḥ-m-ṣ} (\textarabic{ح-م-ص})}
    {Arabic}
    {to toast/roast - referring to the toasted preparation method}
    {The name \textit{M'hamsa} (\textarabic{محمصة}) derives from the Arabic root meaning ``to toast'' or ``to roast.'' The noun form \textit{mḥammaṣa} refers to something that has been toasted---perfectly describing the preparation method of this artisanal pasta.}
    {Originating in Tunisia, particularly in rural and inland regions, M'hamsa reflects a preservation technique born from necessity: turning durum wheat into a shelf-stable, nutrient-dense staple. Its production often took place communally, with women gathering to roll the semolina into irregular pearls and dry them under the sun. This recipe comes from the Silver family of Djerba (David Silver, paternal side), where unlike industrial couscous, m'hamsa is hand-rolled from semolina and water, then sun-dried or lightly toasted to preserve it. The grains are larger and more rustic than standard couscous, roughly the size of peppercorns, with a nutty, toasted flavor that embodies both craftsmanship and resilience in Tunisian food heritage.}
    {MHAMSA.png}

\clearpage

% RIGHT PAGE: Four-Language Ingredients and Instructions
\rightpagecontent
    {%Hebrew ingredients
        \item \texthebrew{בצל}
        \item \texthebrew{פתיתים}
        \item \texthebrew{פפריקה}
        \item \texthebrew{פ.שחור}
        \item \texthebrew{עגבניות}
        \item \texthebrew{ירקות שרוצים}
        \item \texthebrew{מלח}
        \item \texthebrew{מים}
    }
    {%English ingredients
        \item 1 small onion
        \item M'hamsa semolina flakes
        \item Sweet paprika
        \item Black pepper
        \item Tomatoes
        \item Vegetables of choice
        \item Salt
        \item Water
    }
    {%Arabic ingredients
        \item \textarabic{بصلة صغيرة}
        \item \textarabic{فتات السميد (محمصة)}
        \item \textarabic{فلفل أحمر مرحي}
        \item \textarabic{فلفل أكحل}
        \item \textarabic{طماطم}
        \item \textarabic{خضرة حسب الرغبة}
        \item \textarabic{ملح}
        \item \textarabic{ماء}
    }
    {%Spanish ingredients
        \item 1 cebolla pequeña
        \item Granos de sémola m'hamsa
        \item Pimentón dulce
        \item Pimienta negra
        \item Tomates
        \item Verduras al gusto
        \item Sal
        \item Agua
    }
    {%Hebrew instructions
        \item \texthebrew{לטגן בצל קטן}
        \item \texthebrew{להוסיף פתיתים ולאדות}
        \item \texthebrew{להוסיף מים ולבשל}
        \item \texthebrew{לכסות ולבשל עד רך}
    }
    {%English instructions
        \item Sauté chopped onion until translucent
        \item Add m'hamsa and toast lightly
        \item Add water (1¼ cups per cup m'hamsa)
        \item Cover and simmer until tender
    }
    {%Arabic instructions
        \item \textarabic{نقلي البصل حتى يذبل}
        \item \textarabic{نضيف المحمصة ونحمرها}
        \item \textarabic{نصب الماء ونطبخ}
        \item \textarabic{نغطي حتى تطيب}
    }
    {%Spanish instructions
        \item Sofreír cebolla hasta transparente
        \item Añadir m'hamsa y tostar
        \item Agregar agua y cocinar
        \item Tapar hasta que esté tierno
    }

\recipeclosing
    {\texthebrew{בתאבון!}}
    {\textarabic{بالشفا والهناء!}}
    {¡Buen provecho!}
    {Bon Appétit!}
    {From the Silver family kitchen of Djerba, Tunisia}
    {A symbol of Tunisian food heritage, embodying craftsmanship and resilience} 